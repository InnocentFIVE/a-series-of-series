\documentclass[UTF8,15pt,fleqn]{article}
\usepackage{ctex}
\usepackage[utf8]{inputenc}
\usepackage{amssymb}
\usepackage{amsfonts}
\usepackage{amsmath}
\usepackage{upgreek}
\usepackage{graphicx}
\usepackage[b5paper,scale=0.75]{geometry}
\usepackage{unicode-math}
\usepackage{tcolorbox}
\usepackage{fontspec}
\usepackage{tikz}
\usepackage{extarrows}
\usepackage{pxrubrica}
\usepackage{fancyhdr}
\usepackage{titlesec}
\usepackage{ulem}
\usepackage{wallpaper}

\usetikzlibrary{patterns}
\tcbuselibrary{skins} 
\tcbuselibrary{breakable}
\pagestyle{fancy}
\setmainfont{EB Garamond}[Ligatures=Rare]
\setmathfont{Garamond-Math.otf}[StylisticSet={2, 7, 9}]
\setmathfont{Garamond-Math.otf}[StylisticSet={2, 7, 9, 8}, version=a]
\setmathfont[range = "0211C]{Latin Modern Math}
\newcommand{\longrightrightarrows}{\mathrel{\raisebox{-0.43ex}{$\longrightarrow$} \kern-1.498em\raisebox{0.43ex}{$\longrightarrow$}}}
\newcommand{\slongrightrightarrows}{\mathrel{\raisebox{-0.301ex}{$\scriptstyle\longrightarrow$} \kern-1.0486em\raisebox{0.301ex}{$\scriptstyle\longrightarrow$}}}
\newcommand{\snlongrightrightarrows}{\mathrel{\raisebox{-0.301ex}{$\scriptstyle\not\longrightarrow$} \kern-1.0486em\raisebox{0.301ex}{$\scriptstyle\longrightarrow$}}}
\newcommand{\nlongrightrightarrows}{\mathrel{\raisebox{-0.43ex}{$\not\longrightarrow$} \kern-1.498em\raisebox{0.43ex}{$\longrightarrow$}}}
\xeCJKDeclareSubCJKBlock{symbolCJK}{"3001}
\setCJKmainfont[symbolCJK=Source Han Serif SC, BoldFont = GenYoMin TW B]{GenWanMin TW TTF}
\everymath{\displaystyle}
\fancypagestyle{forsection}
{
\fancyhf{}
\renewcommand{\headrulewidth}{0pt}
\fancyfoot[RO]{\vskip-8em\rotatebox[origin=rb]{-90}{{\LARGE\char"2619}\enspace{\bfseries\Roman{page}}\enspace{\LARGE\char"2767}}\kern-3em}
\fancyfoot[LE]{\vskip-8em\hspace{-3em}\rotatebox[origin=lb]{90}{{\LARGE\char"2619}\enspace{\bfseries\Roman{page}}\enspace{\LARGE\char"2767}}}
\fancyfoot[C]{}
\fancyhead[C,L,R]{}
}

\newCJKfontfamily\kaiti{TW-Kai}
\catcode`\,=13
\def,{, }
\catcode`\。=13
\def。{. }
\catcode`\:=13
\def:{: }
\catcode`\;=13
\def;{; }
\catcode`\!=13
\def!{\,!\,}
\catcode`\?=13
\def?{\,?\kern0.5em}
\catcode`\(=13
\def({\raisebox{0.28ex}{\,(}}
\catcode`\)=13
\def){\raisebox{0.28ex}{)\,}}
\catcode`\“=13
\def“{``}
\catcode`\”=13
\def”{''}
\catcode`\、=13
\def、{\mbox{\char"3001\kern-0.35em}}
\catcode`\「=13
\def「{\mbox{\kern-0.35em\char"300C}}
\catcode`\」=13
\def」{\mbox{\char"300D\kern-0.35em}}

\fancyfoot[RO]{\vskip-8em\rotatebox[origin=r]{-90}{{\LARGE ☙}\enspace{\bfseries\Roman{page}}\enspace{\LARGE ❧}}\kern-2.5em}
\fancyfoot[LE]{\vskip-8em\hspace{-2.5em}\rotatebox[origin=l]{90}{{\LARGE ☙}\enspace{\bfseries\Roman{page}}\enspace{\LARGE ❧}}}
\fancyfoot[C]{}
\defaultfontfeatures{RawFeature={+calt,+swsh,+liga,+hlig,+clig,+dlig,+tnum,+lnum}}
\titleformat{\section}[display]
{}{\newpage\thispagestyle{forsection}
	\filright{\hrule width 1cm\vskip-1.95em}\hspace*{1cm}
	\enspace
	{\LARGE\char"E001}
	\enspace
	\raisebox{0.8ex}{$\underset{\mbox{\tiny\textit{SECTION}}}{\textbf{\textit{\Huge{\arabic{section}}}}}$}
	\enspace{\LARGE\char"E002}
	\enspace
	\titlerule}{1em}{\vspace{-3ex}\Large\bfseries\rightline}

\titleformat{\subsection}[hang]
{}{\normalsize\filright\raisebox{-1.3ex}{\huge\char"B6}\kern-2.3ex{\color{white}\normalsize\arabic{subsection}}\enspace}{0.5em}{\normalsize\bfseries}


\begin{document}
\setlength{\lineskip}{5pt}
\setlength{\lineskiplimit}{2.5pt}
\thispagestyle{empty}


\let\symcal\undefined
\DeclareRobustCommand{\symcal}[1]{{\mathpalette\youwillneverusethisnamewhichaboutchi{#1}}}
\newcommand{\youwillneverusethisnamewhichaboutchi}[2]{\mbox{\mathversion{a}$#1\symscr{#2}$}}
% \renewcommand{\symcal}[1]{{\mathversion{a}\mbox{$\symscr{#1}$}}}



\ThisTileWallPaper{\paperwidth}{\paperheight}{cover.pdf}
\phantom{d}
\newpage


\subsection{NONSENCE}

\paragraph{Cauthy收敛定理}
\[
	\forall \varepsilon \left(\exists N\left(\forall n,m>N\left(\left\vert \sum\limits_{m}^{n} a_{\diamond } \right\vert<\varepsilon   \right) \right) \right) \iff \sum a_{\diamond } \longrightarrow a
	.\]

	为了简写记作 \(\sum^{\longrightarrow} a_{\diamond }\),
在这里无穷级数的和定义为部分和的极限:
\[
	\sum a_{\diamond }\coloneq\lim \sum^n a_{\diamond }
	.\]
显然
\[
	\lim a_{\diamond }\neq 0\implies\sum ^{\not\longrightarrow}a_{\diamond } 
	.\]

\begin{itemize}
	\item 比如 \(\sum^{\not\longrightarrow} (-)^{\diamond }\) 。
\end{itemize}

另,级数是否收敛只取决与无穷远处的形状,与有限处无关。

\subsection{审敛}

{\kaiti 为了方便记忆,以下审敛法都写成极限形式,这意味着判别力比原来要弱。}


\paragraph{比较判别}
若\(	|a_n| = \symcal O(|b_n|)\),
则:
\[
	\begin{aligned}
		\sum^{ \not\longrightarrow} |a_\diamond|\implies \sum^{ \not\longrightarrow} |b_\diamond| , \\
		\sum ^{\longrightarrow }|b_\diamond| \implies \sum^{ \longrightarrow} |a_\diamond| .
	\end{aligned}
\]

\paragraph{ \(p\) 级数法}

\[
	\sum^\longrightarrow \frac{1}{\diamond^n },\, \text{iff}\ \Re(n)>1 ,\,\text{otherwise\ }\sum^{\not\longrightarrow } \frac{1}{\diamond^n }
	.\]

\paragraph{D'Alembert审敛}

若
\[
	\begin{aligned}
		 & \limsup\left|\frac{a_{\diamond +1}}{a_\diamond }\right| <1 \implies\sum ^{ \longrightarrow}a_\diamond, \\
		 & \liminf\left|\frac{a_{\diamond +1}}{a_\diamond }\right| >1 \implies\sum^{\not\longrightarrow} a_\diamond. \\
		 & \mbox{\kaiti 否则不能判定}  .
	\end{aligned}
\]

\paragraph{Cauthy审敛}
若
\[
	\begin{aligned}
		 & \limsup\left| \sqrt[\diamond ]{a_\diamond }\right|<1\implies \sum^{\longrightarrow} a_\diamond ,  \\
		 & \liminf\left| \sqrt[\diamond ]{a_\diamond }\right| >1 \implies\sum^{\not\longrightarrow} a_\diamond, \\
		 & \mbox{\kaiti 否则不能判定}  .
	\end{aligned}
\]

\paragraph{Raabe审敛}

若
\[
	\begin{aligned}
		 & \lim \diamond \left(\left|  \frac{a_\diamond }{a_{\diamond +1}}\right|-1 \right) > 1 \implies\sum^{\longrightarrow} a_\diamond  , \\
		 & \lim \diamond \left(\left|  \frac{a_\diamond }{a_{\diamond +1}}\right|-1 \right) < 1 \implies\sum ^{\not\longrightarrow}a_\diamond,  \\
		 & \mbox{\kaiti 否则不能判定}  .
	\end{aligned}
\]

\paragraph{积分判别法}

\[
	\forall x\geqslant 1	(f(x)\searrow \land f(x)>0) \implies \sum f(\diamond )\,\mbox{\kaiti 与}  \, \int_1^\infty f(x)\,\mathrm{d}x \,\mbox{\kaiti 共敛散}
	.\]


\paragraph{Leibniz审敛}

\[
	\forall x\geqslant 1	(f(x)\searrow \land f(x)>0\land \lim f(x)=0) \implies \sum  ^\longrightarrow (-)^\diamond f(\diamond ) 
	.\]

\paragraph{Dirichlet审敛}
若 \(a_\diamond \)单调,则:
\[
\exists M\left( \left\vert \sum^n b_\diamond  \right\vert <M(\lim a_\diamond =0)\right) \implies \sum^\longrightarrow   a_\diamond b_\diamond  
.\]

\paragraph{Abel审敛}

若 \(a_\diamond \)单调,\(\sum ^{\longrightarrow}b_\diamond \) ,则:
\[
\exists M(a_\diamond <M)\implies\sum^{\longrightarrow} a_\diamond b_\diamond  .
\]

\paragraph{Ermakov审敛}

\[
\begin{aligned}
&\exists F(\forall x>F(f(x)\searrow\land f(x)>0))\land\exists G(\forall x>G(g(x)\nearrow\land g(x)>\land g(x)\in\symcal C^1)) \\ 
&\implies\exists X\left(\forall x>X\left( \begin{aligned}
 &\frac{f(g)g'}{f}\leqslant q<1\implies\sum^\longrightarrow f(\diamond ) \\ 
&\frac{f(g)g'}{g} \geqslant 1\implies\sum^{\not\longrightarrow}f(\diamond )
\end{aligned} \right)  \right) .
\end{aligned}
\]

\paragraph{Cauthy的凝聚判别}

\[
\exists N(n>N(a_n\searrow\land a_n\geqslant 0))\implies \sum a_\diamond\,\mbox{\kaiti 与}  \,\sum 2^\diamond a_{2^\diamond } \,\mbox{\kaiti 共敛散} 
.\]
\subsection{绝对与条件收敛}
\[
	\begin{aligned}
		 & \sum^\longrightarrow |a_\diamond |\eqcolon \sum a_\diamond  \mbox{\kaiti 绝对收敛}  ,                                  \\
		 & \sum^{\not\longrightarrow} |a_\diamond |\land \sum^\longrightarrow a_\diamond  \eqcolon \sum a_\diamond  \mbox{\kaiti 条件收敛}.
	\end{aligned}
\]

绝对收敛的级数可以任意改变运算顺序,但条件收敛不可以。

\subsection{函数项级数}

\paragraph{一致收敛}

\[
\forall \varepsilon >0(\forall x\in\symcal{I} (\exists N(\forall n,m>N(\left\vert f_n(x)-f_m(x) \right\vert<\varepsilon  ))))\eqcolon \sum f_\diamond (x)\mbox{\kaiti 在 \(\symcal{I} \)上一致收敛}  
.\]

记作 \( \sum f_\diamond (x) \longrightrightarrows f(x)\)。

一致收敛的级数可以交换求和与积分,求导,极限。显然连续函数项的级数若一致收敛,则和函数必然连续。一致收敛是对整个区间而言的,比普通的逐点收敛强。

显然,若 \(\displaystyle f_\diamond(x) \nlongrightrightarrows 0 \),则 \(\sum^{\snlongrightrightarrows} f_\diamond (x)\)。

\paragraph{内闭一致收敛}

实际上就是在开区间 \(\symcal{I} \) 内的任意一个闭区间上都一致收敛。
\begin{itemize}
	\item 比如 \(\sum \diamond \exp (-\diamond ^2x)\) 在 \((0,\infty)\) 上不一致收敛,具体是靠近 \(x=0\) 时收敛速度无限衰减,但在 \((0,\infty)\) 上内闭一致收敛。
\end{itemize}

\paragraph{Weierstrass判别法}
\[
	\sum^\longrightarrow\sup_{x \in\symcal{I} }|f_\diamond (x)| \implies\sum^{\slongrightrightarrows}_{x\in\symcal I} f_\diamond (x)  
.\]

\paragraph{Dirichlet判别}

若 \(f_n(x)\) 对 \(n\) 单调
\[
\exists M\left(\forall n\left(\forall x\in\symcal{I} \left(\left|\sum^n g_\diamond (x)\right|<M\right)\right)\right)\land f_\diamond (x)\longrightrightarrows 0\implies\sum^{\slongrightrightarrows} f_\diamond(x) g_\diamond (x)
.\]

其中我们称 \(\exists M\left(\forall n\left(\forall x\in\symcal{I} \left(\left|\sum^n g_\diamond (x)\right|<M\right)\right)\right)\) 为 \(g_\diamond (x)\) 的部分和在 \(\symcal{I} \) 上一致有界。

\paragraph{Abel 判别}

若 \(f_n(x)\) 对 \(n\) 单调且在  \(\symcal{I} \) 上一致有界,则:
\[
\sum^{\slongrightrightarrows} b_\diamond (x)\implies\sum^{\slongrightrightarrows} f_\diamond(x) g_\diamond (x)
.\]
\end{document}


